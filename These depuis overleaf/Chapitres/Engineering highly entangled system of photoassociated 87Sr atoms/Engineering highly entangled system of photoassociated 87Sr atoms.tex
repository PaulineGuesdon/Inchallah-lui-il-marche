% !TeX root = ../../main.tex
\chapter{Engineering highly entangled system of photoassociated 87Sr atoms}
Engineering Dicke states

\section{Introduction on photoassociation}
\subsection{Preamble on scattering theory}

It is convenient in the case of the $^{87}Sr$ to approximate the electrons following the movement
of the nuclei because of the mass ratio. In a Born-Oppenheimer picture we neglect the coupled
terms between the kinetic energy of the electrons and the kinetic energy of the nuclei.
The effective hamiltonian is given by

\begin{equation}
    \hat{H} = -\frac{\hbar^2}{2\mu}\left(\frac{1}{r^2} \frac{d^2u}{dr^2}r^2 - \frac{l(l+1)}{r^2} 
    + \hat{V}(r)\right)u(r) = E u(r)
\end{equation}
$u(r)$ the eigenfunction of the hamiltonian $\hat{H}$.\\

% $\hat{V}(r)$ are higher order electronic interactions equals to $-C_n/r^n$. 
% To solve this hamiltonian it is easier to change the basis of study as in 
% \cite{dalibard_interactions_nodate}.
% The major contribution to the potential are $3$rd and $6$th order terms rspectively the dipole-
% dipole and the van der Waals interaction. \\
% At really long distance the interaction that counts is the van der Waals. The neutral atoms
% do not feel any effect on their electronic shell.\\
% For closer distance $r$ of the order of the de De Broglie wavelength the force that counts 
% is the kinetic energy over the potential and centrifugal force. \\
% For medium distance $r_1$ the electronic shell start to overlap and the coulomb force is dominant.\\
% For very short distance around the turning point of the potential higher order terms only
% count ($1/r^2$ and $V(r)$) and the potential is repulsive.\\

The solution for $r \sim r_1$ potential is a superposition of incoming wave and a scattered one that 
acquire a phase $\delta_l$

\begin{equation}
    \psi(r) =  sin(kr - l\frac{\pi}{2}) + tan(\delta_l) cos(kr - l\frac{\pi}{2})
\end{equation}
At this position $r_1 = a$ where the potential appears as a scattering boundary this distance 
is called the scattering length. 

Continuity of the wavefunction in every region should be respected. It gives 
 \begin{equation}
    tan(\delta_l) \propto k^{2l+1}
 \end{equation}

 For s-waves ($l=0$), this express as 
\begin{equation}
    a = -\lim_{k\to 0} tan(\delta_0)/k
\end{equation}

A scattering cross section is usually defined from this physical quantity
as the area of a disk with a radius equal to the scattering length, which 
gives the formula 
\begin{equation}
    \sigma = 4\pi a^2
\end{equation}

In ultracold gases, only s-wave atoms can collide because they do not have 
enough kinetic energy to pass the potential barrier of higher angular momentum states. It imposes
that for fermions -by parity of the total wavefunction being antisymetric- 
only atoms with even angular momentum can collide meaning their spin is in a singlet state as 
in the case of $^{87}Sr$. For bosons it is the contrary : from an even orbital wavefunction,
the spin wavefunction should be antisymetric to collide.

\subsection{New try to rely optical length to scattering length}
It is convenient in the case of the $^{87}Sr$ to approximate the electrons following the movement
of the nuclei because of the mass ratio. In a Born-Oppenheimer picture we neglect the coupled
terms between the kinetic energy of the electrons and the kinetic energy of the nuclei. In this 
manner we can study separately the radial and the angular part of the electronic movement. \\
As in the LANDAU-ZENER we will find the solutions of the 1D radial Schrödinger equation of the 
electrons by approximating the scattering of the atoms as a sum of an incoming plane wave with a 
spherical scattered one - considered as plane wave at long distance - 

\begin{equation}
    \Psi = e^{i\mathbf{k}\cdot \mathbf{z}} + f(\theta) \frac{e^{ikr}}{r}
\end{equation}
$f(\theta)$ a function that will show the inelasticity and/or elasticity part of the scattering
by the absorption or not of the wave. $theta$ is the angle between the vector wore by the 
incoming wave with the vector wore by the scattered wave.\\

\subsection{What is photoassociation}
\begin{figure}[H]
    \centering
    \includegraphics[width=0.7\linewidth]{Chapitres/Engineering highly entangled system of photoassociated 87Sr atoms/PA_concept.png}
    \caption{Caption}
    \label{fig:PA_concept}
\end{figure}

Photoassociating two atoms consists in bounding two colliding atoms with light 
that occurs mostly with two-body and three body losses. 

As visible in \ref{fig:PA_concept} from a free state of a two-atoms system we couple them via a laser with a molecular 
vibrationnal state. In the case of spectroscopy we usually start with the least bounded state to have
the atomic state reference. From that we sweep the atom frequency to adress the
molecular states red detuned from the atomic line. 
The quantity that will testify there is two-body losses
is the atom number remaining in the trap after a photoassociation pulse. 
The molecule leaves the trap by acquiring kinetic energy from the binding energy.
The experimental proof of photoassociation is given by 
\begin{equation}
    \frac{dN}{dt} = - \beta N^2 - \Gamma N
\end{equation}

$\beta$ the fitted parameter of two-body losses ($\beta = \frac{K_2 V_e}{2}$,
$K_2$ in comparison with \cite{han_photoassociation_2018}) the two-body 
loss rate, $V_e$ the effective volume of the cloud $(2\pi )^{3/2}\sigma_x 
\sigma_y \sigma_z$ for a density Gaussian distribution and $\Gamma$ 
the one-body losses rate. This latter is limited by the vacuum quality and 
the lifetime of the atoms in the trap. 

PA is mostly used in Feschbach resonance field because determining the exact 
position of the vibrational states enables by changing the magnetic field
to tune the scattering length of the atoms and thus the interactions in the system.

\subsection{Molecular formalism/vocabulary (condon radius, optical length...)}
\textbf{Optical length} characterizes the strength of the photoassociation rate 

\begin{equation}
    K_2= \frac{4\pi \hbar}{\mu} nl_{opt}
\end{equation}

$\mu$ the reduced mass of the two atoms. This rate represents
the ratio of the atoms that are lossed by 2-body losses. \\

It contains the probability to transition from an initial vibrational $i$ 
state to a final vibrational $f$ state : \textbf{the Franck Condon factor} (\cite{guttridge_photoassociation_2019})

\begin{equation} 
    F_{FC} = \Bigg|\int_0^{\infty} \psi_i(r) \psi_f(r) dr\Bigg |^2 
\end{equation}

r the inter-nuclear distance. 
It depends directly with the overlap of the wavefunctions of the initial and final states.
\textbf{The Condon radius} is the distance between the atoms for which this factor is maximum
which also means that in a classical approach the atoms spend the most time at this position.\\


\textbf{The good quantum numbers}\\
We write the total angular momentum of the molecule $T = R + F = R + I + J$, $F$ being the 
total spin of the two atoms $f_1 + f_2$, $J = j_1+j_2$ is the angular momentum
of the molecule that describes the spin-orbit coupling, and R the rotational angular momentum of the molecule. \\

We define the projection of the different angular momentum on the inter-nuclear axis as
\begin{equation}
    \Omega = \Lambda + \Sigma
\end{equation}

with $\Lambda$ the projection of the orbital momentum of the molecule on the internuclear axis
and $\Sigma$ the projection of the momentum spin.

$M_T$ is the projection of the total angular momentum $T$ onto a defined quantization axis.\\

For the rotational quantum number there is a parity of the total wavefunction that needs to 
be fulfilled. In addition there are only s-wave collisions in cold gases then the part of the 
wavefunction should be odd the spin one antisymetric
\begin{equation}
    \ket{\Psi} = \left(\ket{\phi_e}_A \ket{\phi_g}_B + (-1)^R \ket{\phi_e}_B \ket{\phi_g}_A\right)
     \otimes\ket{\chi}
\end{equation}
$\ket{\phi_e}$ and $\ket{\phi_g}$ are the electronic wavefunctions of the two atoms A and B 
in ground and excited state respectively, $\ket{\chi}$ is the spin wavefunction.
It gives accessible values $R = 0, 2$ or $4$...

For $^1S_0$ - $^3P_1$ molecular state there is a strong spin-orbit coupling that couples its 
related momentum to the internuclear axis (\ref{fig:hund}), quantified by $\Omega$ and described as Hund's case 
(c) \\

\begin{figure}[H]
    \centering
    \includegraphics[width=0.5\linewidth]{Chapitres/Engineering highly entangled system of photoassociated 87Sr atoms/Hund_case_c.png}
    \caption{hund}
    \label{fig:hund}
\end{figure}

The experiment presented in the following section is photoassociation in the $F=11/2$ hyperfine
state. We will focus in this state to simplify the discussion.\\
Say we have one atome in the $^1S_0$ state and one in the $^3P_1$ state: we can have a total spin of $J = 0, 1$ 
or $2$. It gives possible values of total atomic angular momentum F:

\begin{align}
\text{For } J=0, &\quad F=\frac{9}{2} \\
\text{For } J=1, &\quad F=\frac{7}{2},\frac{9}{2},\frac{11}{2} \\
\text{For } J=2, &\quad F=\frac{5}{2},\frac{7}{2},\frac{9}{2},\frac{11}{2},\frac{13}{2}
\end{align}\\


We have one atom with $f_1=9/2$ from the $^1S_0$ state and one in the $^3P_1$ state with $f_2=11/2$.
The possible values of total angular momentum of the molecule are 
\begin{align}
\text{For } R=0, &\quad {T = 1, 2, 3... 10}\\
\text{For } R=2, &\quad {T = 1,2... 12}\\
\text{For } R=4, &\quad {T = 1,2 ... 14}
\end{align}

The initial $T^i = i_1+i_2$ with $R=0$ because we are in s-wave collisions. In respect with
the parity $p = +1$ the possible values of $T^i$ are $T^i = 0, 2, 4...$.\\
As for atoms there are selection rules for transitions from a free state to a molecular state
fixed by the angular momentum of the electrons. 
$\Delta J = 0, 1$ gives $\Delta T = 0, 1$ with $T=0 \rightarrow T=0$ forbidden.\\
In the end the accessible transitions in the first molecular excited state of the $f_1=11/2$ are
\begin{align}
\text{For } R=0, &\quad {T^f = 1, 2, 3... 11}\\
\text{For } R=2, &\quad {T^f = 1,2... 13}\\
\text{For } R=4, &\quad {T^f = 1,2 ... 15}\\
...
\end{align}\\
There are 981 combinations of accessible $(R,T^f)$ molecular states\\
Another element to describe the molecule is the symmetry of the orbital wavefunction. For an
even wavefunction the exchange of the two atoms $A$ and $B$ does not change the sign of the 
wavefunction, as in left figures \ref{fig:u-g symmetry}. We say the potential is \textbf{gerade}.
The probability of finding electrons in between the two atoms is high compared to the 
\textbf{ungerade} potential where the odd wavefunction cancels the probability of finding electrons in between.

\begin{figure}[H]
    \centering
    \includegraphics[width=1\linewidth]{Chapitres/Engineering highly entangled system of photoassociated 87Sr atoms/gerade_ungerade.png}
    \caption{u-g symmetry}
    \label{fig:u-g symmetry}
\end{figure}

\subsection{External energy states}
As seen in the subsection 3.1.1 in a molecular picture the hamiltonian express as 
in \cite{han_photoassociation_2018} by 

\begin{equation}
    \hat{H} = \frac{\hat{p_r}^2}{2\mu} + \frac{\hbar^2}{2\mu r^2}R\left(R+1\right) +\hat{V}_{BO} + \hat{H}_{HF} 
\end{equation}
with
\begin{equation}
    \hat{V}_{BO} = \frac{-C_6}{r^6}\left(1-\frac{\sigma^6}{r^6}\right) - s\frac{C_3^{\Omega}}{r^3}
\end{equation}

\begin{equation}
\hat{H}_{\mathrm{hf}} =
A \, (\mathbf{i}_1 \cdot \mathbf{j}_1)
+
B \,
\frac{
3(\mathbf{i}_1 \cdot \mathbf{j}_1)^2
+ \frac{3}{2}(\mathbf{i}_1 \cdot \mathbf{j}_1)
- i_1(i_1+1)\, j_1(j_1+1)
}{
2 i_1 (2 i_1 - 1)\,
2 j_1 (2 j_1 - 1)
}
\end{equation}

$s=\pm1$ for gerade and ungerade potentials respectively.
The hyperfine term of the atom in $^1S_0$ is zero because its electronic angular momentum is zero.
Only the hyperfine interaction of the atom with $i_1, j_1$ in $^3P_1$ counts. 
\subsubsection{WKB approximation}
\begin{figure}[H]
    \centering
    \includegraphics[width=0.7\linewidth]{Chapitres/Engineering highly entangled system of photoassociated 87Sr atoms/sakurai_WKB.png}
    \caption{Caption}
    \label{fig:pa-sakurai-wkb}
\end{figure}

\subsubsection{}
\subsection{Internal energy states}
Leroy-Bernstein approx
\section{About photoassociation on other species}

\subsection{Mass scaling ($^{88}Sr$)}
$^{88}Sr$ is a bosonic isotope of strontium with no hyperfine structure. The photoassociation
spectrum is then simpler than the fermionic ones : only two potentials in the first excited
molecular state $0_u$ and $1_u$. Results have been presented in \cite{zelevinsky_narrow_2006} 
where their fitting model gives the following potential expression
\begin{equation}
    \hat{V}_{BO} \sim \frac{-C_{6,\Lambda}}{r^6} -\frac{C_3}{r^3} + \frac{C_J}{r^2}
\end{equation}
for the two accessible potentials $\Lambda = 0, 1$. 

Mass scaling method presented in \cite{reschovsky_narrow-line_2018} enables to predict the position
of the vibrational states of the different istopoes of strontium. \\

\begin{figure}[H]
    \centering
    \includegraphics[width=0.7\linewidth]{Chapitres/Engineering highly entangled system of photoassociated 87Sr atoms/mass_scaling.png}
    \caption{Caption}
    \label{fig:pa-mass-scaling}
\end{figure}

\subsection{Ytterbium: hfs}
As the $^{87}Sr$ the fermionic \textbf{$^{173}Yb$} has a hyperfine structure that makes its molecular 
potential landscape more complex than many known species. 


\begin{figure}[H]
    \centering
    \includegraphics[width=0.7\linewidth]{Chapitres/Engineering highly entangled system of photoassociated 87Sr atoms/Ytterbium_molecular_potentials.png}
    \caption{Caption}
    \label{fig:pa-ytterbium-potentials}
\end{figure}

\textbf{$^{171}Yb$}
In their fitting model they consider the same interactions 
as in our case (expressed in equation (3.22)) but without the hyperfine term.
\section{Experimental setup}

\section{88Sr Results}
Lopt, power broadening,  thermal broadening...

\begin{figure}[H]
    \centering
    \includegraphics[width=0.7\linewidth]{Chapitres/Engineering highly entangled system of photoassociated 87Sr atoms/88Sr_molecular_potentials.png}
    \caption{Caption}
    \label{fig:pa-88sr-potentials}
\end{figure}

\subsection{Technical issues of inhabilitation of photoassociation}
\subsubsection{Laser width}

\section{87Sr molecules}

Lopt
questions sur nb quantique / choix de pompage optique

\subsection{Physical sources of inhabilitation of photoassociation}
\subsubsection{On F = 9/2 : predissociation}
\subsubsection{Coupling to more energetic state from the IR}
\subsubsection{Node of wavefunction for some vibrational states}
\subsection{Energy landscape of 87Sr-87Sr molecules}
