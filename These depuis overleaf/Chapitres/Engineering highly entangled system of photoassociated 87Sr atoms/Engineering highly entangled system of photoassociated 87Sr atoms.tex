\chapter{Engineering highly entangled system of photoassociated 87Sr atoms}
Engineering Dicke states

\section{Introduction on photoassociation}

\subsection{What is photoassociation}
\begin{figure}[H]
    \centering
    \includegraphics[width=0.7\linewidth]{Chapitres/Engineering highly entangled system of photoassociated 87Sr atoms/PA_concept.png}
    \caption{Caption}
    \label{fig:PA_concept}
\end{figure}

Photoassociating two atoms consists in bounding two colliding atoms with light 
that occurs mostly with two-body and three body losses. 

The first necessity for photoassociating atoms is collisions. Each species has a 
specific cross section for which particles can collide with another, which is defined
as

\begin{equation}
    \sigma = 4\pi a^2
\end{equation}

with $a$ the scattering length we will describe in the next section. 

In ultracold gases, only s-wave atoms can collide because they do not have 
enough kinetic energy to pass the potential barrier of higher angular momentum states. It imposes
that for fermions -by parity of the total wavefunction being antisymetric- 
only atoms with even angular momentum can collide meaning their spin is in a singlet state as 
in the case of 87Sr. For bosons it is the contrary : from an even orbital wavefunction,
the spin wavefunction should be antisymetric to collide.

The second element for PA is a resonant laser. As visible in \ref{fig:PA_concept} : 
from a free state of a two-atoms system we couple them via a laser with a molecular 
vibrationnal state. In the case of spectroscopy we usually start with the least bounded state to have
the atomic state reference. From that we sweep the atom frequency to adress the
molecular states red detuned from the atomic line. 

PA is mostly used in Feschbach resonance field because determining the exact 
position of the vibrational states enables by changing the magnetic field
to tune the scattering length of the atoms and thus the interactions in the system.



\subsection{Molecular formalism/vocabulary (condon radius, optical length...)}
Optical length is defined as the imaginary part of the scattering length and 
describes the strength of the photoassociation rate 

\begin{equation}
    K_2= \frac{4\pi \hbar}{\mu} nl_{opt}
\end{equation}

$n$ the volumic density of the cloud, $\mu$ the reduced mass of the two atoms. This rate represents
the ratio of the atoms that are lossed by 2-body losses.\\

The Franck Condon factor is the probability to transition from an initial vibrational $i$ state to 
a final vibrational $f$ state

\begin{equation} 
    F_{FC} = \Bigg|\int_0^{\infty} \psi_i(R) \psi_f(R) dR\Bigg |^2 
\end{equation}


R the inter-nuclear distance.
It depends directly with the overlap of the wavefunctions of the initial and final states.
The Condon radius is the distance between the atoms for which this factor is maximum
which also means that in a classical approach the atoms spend the most time at this position.\\


\textbf{The good quantum numbers}\\
We write the total angular momentum of the molecule $T = R + F = R + I + J$, $F$ being the 
total spin of the two atoms $f_1 + f_2$, $J = j_1+j_2$ is the angular momentum
of the molecule that describes the spin-orbit coupling, and R the rotational angular momentum of the molecule. \\

We define the projection of the different angular momentum on the inter-nuclear axis as
\begin{equation}
    \Omega = \Lambda + \Sigma
\end{equation}

with $\Lambda$ the projection of the orbital momentum of the molecule on the internuclear axis
and $\Sigma$ the projection of the momentum spin.

$M_T$ is the projection of the total angular momentum $T$ onto a defined quantization axis,
sometimes being the internuclear axis depending on the Hund's case.\\

For the rotational quantum number there is a parity of the total wavefunction that needs to 
be fulfilled. In addition there are only s-wave collisions in cold gases then the part of the 
wavefunction should be odd the spin one antisymetric
\begin{equation}
    \ket{\Psi} = \left(\ket{\phi_e}_A \ket{\phi_g}_B + (-1)^R \ket{\phi_e}_B \ket{\phi_g}_A\right)
     \otimes\ket{\chi}
\end{equation}
$\ket{\phi_e}$ and $\ket{\phi_g}$ are the electronic wavefunctions of the two atoms A and B 
in ground and excited state respectively, $\ket{\chi}$ is the spin wavefunction.
It gives accessible values $R = 0, 2$ or $4$...

The hamiltonian of the homonuclear molecule 
For $^1S_0$ - $^3P_1$ molecular state there is a strong spin-orbit coupling that couples its 
related momentum to the internuclear axis (\ref{fig:hund}), quantified by $\Omega$ and described as Hund's case 
(c) \\

\begin{figure}[H]
    \centering
    \includegraphics[width=0.5\linewidth]{Chapitres/Engineering highly entangled system of photoassociated 87Sr atoms/Hund_case_c.png}
    \caption{hund}
    \label{fig:hund}
\end{figure}

The experiment presented in the following section is photoassociation in the $F=11/2$ hyperfine
state. We will focus in this state to simplify the discussion.\\
Say we have one atome in the $^1S_0$ state and one in the $^3P_1$ state: we can have a total spin of $J = 0, 1$ 
or $2$. It gives possible values of total atomic angular momentum F:

\begin{align}
\text{For } J=0, &\quad F=\frac{9}{2} \\
\text{For } J=1, &\quad F=\frac{7}{2},\frac{9}{2},\frac{11}{2} \\
\text{For } J=2, &\quad F=\frac{5}{2},\frac{7}{2},\frac{9}{2},\frac{11}{2},\frac{13}{2}
\end{align}\\


We have one atom with $f_1=9/2$ from the $^1S_0$ state and one in the $^3P_1$ state with $f_2=11/2$.
The possible values of total angular momentum of the molecule are $F = 1, 2, 3, 4, 5, 6, 7, 8, 9$ or $10$.


\subsection{External energy states}
To describe two-body losses of a photoassociation process, we can change the study basis 
depending on the distance between the two atoms by approximating dominant couplings 
over others. (Hund's cases) \\
The hamiltonian of the two-atom system is given by

\begin{equation}
    \hat{H} = \hat{T} + \hat{V}_{BO} + \frac{\hbar^2}{2\mu r^2} + \hat{H}_{HF} 
\end{equation}
with
\begin{equation}
    \hat{V}_{BO} = \frac{-C_6}{r^6}\left(1-\frac{\sigma^6}{r^6}\right) - s\frac{C_3^{\Omega}}{r^3}
\end{equation}

\subsubsection{WKB approximation}
\begin{figure}[H]
    \centering
    \includegraphics[width=0.7\linewidth]{Chapitres/Engineering highly entangled system of photoassociated 87Sr atoms/sakurai_WKB.png}
    \caption{Caption}
    \label{fig:placeholder}
\end{figure}

\subsubsection{}
\subsection{Internal energy states}
Leroy-Bernstein approx
\section{About photoassociation on other species}

\subsection{Mass scaling (88Sr)}
\begin{figure}[H]
    \centering
    \includegraphics[width=0.7\linewidth]{Chapitres/Engineering highly entangled system of photoassociated 87Sr atoms/mass_scaling.png}
    \caption{Caption}
    \label{fig:placeholder}
\end{figure}

\subsection{Ytterbium: hfs}

\begin{figure}[H]
    \centering
    \includegraphics[width=0.7\linewidth]{Chapitres/Engineering highly entangled system of photoassociated 87Sr atoms/Ytterbium_molecular_potentials.png}
    \caption{Caption}
    \label{fig:placeholder}
\end{figure}

\section{Experimental setup}

\section{88Sr Results}
Lopt, power broadening,  thermal broadening...

\begin{figure}[H]
    \centering
    \includegraphics[width=0.7\linewidth]{Chapitres/Engineering highly entangled system of photoassociated 87Sr atoms/88Sr_molecular_potentials.png}
    \caption{Caption}
    \label{fig:placeholder}
\end{figure}

\subsection{Technical issues of inhabilitation of photoassociation}
\subsubsection{Laser width}

\section{87Sr molecules}

Lopt
questions sur nb quantique / choix de pompage optique

\subsection{Physical sources of inhabilitation of photoassociation}
\subsubsection{On F = 9/2 : predissociation}
\subsubsection{Coupling to more energetic state from the IR}
\subsubsection{Node of wavefunction for some vibrational states}
\subsection{Energy landscape of 87Sr-87Sr molecules}
