\chapter{Engineering highly entangled system of photoassociated 87Sr atoms}
Engineering Dicke states

\section{Introduction on photoassociation}

\subsection{What is photoassociation}
\begin{figure}[H]
    \centering
    \includegraphics[width=0.7\linewidth]{Chapitres/Engineering highly entangled system of photoassociated 87Sr atoms/PA_concept.png}
    \caption{Caption}
    \label{fig:PA_concept}
\end{figure}

Photoassociating two atoms consists in bounding two colliding atoms with light 
that occurs mostly with two-body and three body losses. 

The first necessity for photoassociating atoms is collisions. Each species has a 
specific cross section for which particles can collide with another, which is defined
as

\begin{equation}
    \sigma = 4\pi a^2
\end{equation}

with $a$ the scattering length we will describe in the next section. 

In ultracold gases, only s-wave atoms can collide because they do not have 
enough kinetic energy to pass the potential barrier of higher angular momentum states. It imposes
that for fermions -by parity of the total wavefunction being antisymetric- 
only atoms with even angular momentum can collide meaning their spin is in a singlet state as 
in the case of 87Sr. For bosons it is the contrary : from an even orbital wavefunction,
the spin wavefunction should be antisymetric to collide.

The second element for PA is a resonant laser. As visible in \ref{fig:PA_concept} : 
from a free state of a two-atoms system we couple them via a laser with a molecular 
vibrationnal state. In the case of spectroscopy we usually start with the least bounded state to have
the atomic state reference. From that we sweep the atom frequency to adress the
molecular states red detuned from the atomic line. 

PA is mostly used in Feschbach resonance field because determining the exact 
position of the vibrational states enables by changing the magnetic field
to tune the scattering length of the atoms and thus the interactions in the system.



\subsection{Molecular formalism/vocabulary (condon radius, optical length...)}

\subsection{Internal energy states}
\subsubsection{WKB approximation}
\begin{figure}[H]
    \centering
    \includegraphics[width=0.7\linewidth]{Chapitres/Engineering highly entangled system of photoassociated 87Sr atoms/sakurai_WKB.png}
    \caption{Caption}
    \label{fig:placeholder}
\end{figure}
\subsubsection{}
\subsection{External energy states}
\section{About photoassociation on other species}

\subsection{Mass scaling (88Sr)}
\begin{figure}[H]
    \centering
    \includegraphics[width=0.7\linewidth]{Chapitres/Engineering highly entangled system of photoassociated 87Sr atoms/mass_scaling.png}
    \caption{Caption}
    \label{fig:placeholder}
\end{figure}

\subsection{Ytterbium: hfs}

\begin{figure}[H]
    \centering
    \includegraphics[width=0.7\linewidth]{Chapitres/Engineering highly entangled system of photoassociated 87Sr atoms/Ytterbium_molecular_potentials.png}
    \caption{Caption}
    \label{fig:placeholder}
\end{figure}

\section{Experimental setup}

\section{88Sr Results}
Lopt, power broadening,  thermal broadening...

\begin{figure}[H]
    \centering
    \includegraphics[width=0.7\linewidth]{Chapitres/Engineering highly entangled system of photoassociated 87Sr atoms/88Sr_molecular_potentials.png}
    \caption{Caption}
    \label{fig:placeholder}
\end{figure}

\subsection{Technical issues of inhabilitation of photoassociation}
\subsubsection{Laser width}

\section{87Sr molecules}

Lopt
questions sur nb quantique / choix de pompage optique

\subsection{Physical sources of inhabilitation of photoassociation}
\subsubsection{On F = 9/2 : predissociation}
\subsubsection{Coupling to more energetic state from the IR}
\subsubsection{Node of wavefunction for some vibrational states}
\subsection{Energy landscape of 87Sr-87Sr molecules}
